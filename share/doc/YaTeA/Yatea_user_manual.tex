% Author:  Jaakko L�fstr�m 10/1/2005
% mods by Wray

% note we have pretty much forced document style "article" and pagination details
% on you, add your EPSF support here or whatever

% note we have no place for confidentiality considerations, so please
% place this in bold in your abstract too
\documentclass[a4paper]{article}
%\usepackage{alvis}
\usepackage{listings}
\usepackage{moreverb}
\usepackage{listings}
\usepackage{graphicx}

\title{\YaTeA\\User manual}
\author{Sophie Aubin, Thierry Hamon}
\lstset{language=XML}
\lstset{basicstyle=\footnotesize, }

   \def\YaTeA{{\rmfamily Y\kern-.36em%
    \lower.7ex\hbox{A}\kern-.25em%
    T\kern-.1667em\lower.7ex\hbox{E}\kern-.08emA}~}

\begin{document}
\maketitle

\section{Introduction}
\label{Introduction}


\YaTeA  aims at identifying and extracting phrases
which are potential terms ({\em i.e. } term candidates). Moreover each
term candidate is syntactically analysed in order to identify head and
modifier components. 

Data used for the identification and syntactic analysis of the term
candidates are provided. They can be modified by
the user. New data can also be defined by the user. 

Different kinds of outputs are offered, among which, 
\begin{itemize}
\item a list of term candidates in a XML format,
\item an XML annotation of the corpus with either testified terms or
  term candidates in a format compatible with the biolg version of
  the Link Grammar Parser,
\item a list of term candidates in text format
\item the HTML visualisation of term candidates and non extracted phrases
  marked in the corpus 
\end{itemize}


\YaTeA takes a corpus in the TreeTagger format as an input.



\section{Running \YaTeA}
\label{RunningYatea}


The term extractor can be run according to a command line:

\sloppy
\texttt{perl yatea.pl file\_in.ttg -language \textit{language-code} [-rcfile
  \textit{configuration-file}] [-termino  \textit{file}] [-match-type \textit{loose} or \textit{strict}]
  [output-style-options]
  % [-MNPhtml] [-printChunking] %[-compareChunking
                                %\textit{file}] 
  %[-xmlout] [-suffix] %[-correction] [-annotate] 
  %[-TTforLGP] [-TCforLGP] [termList] %[basicTerms]
  }


\texttt{file\_in.ttg} is the input corpus in the TreeTagger format.
\texttt{[output-style-options]} can be chosen among the options
described below. Several can be specified. 

\textbf{Options} : 
\begin{itemize}
\item \texttt{-help} : displays the message about command line and
  options

\item \texttt{-language} \textit{langue-code} : (mandatory) definition of the
  corpus language (\texttt{EN} for English, \texttt{FR} for
  French). This information allows the selection of the right configuration files.

\item \texttt{-rcfile} : (optional) specification of a particular
  configuration file containing pathes to repositories containing
  linguistic data files. 

\item \texttt{-suffix} : (optional) specification of a name for the current version of the analysis. results are gathered in a specific directory of this name and result files also carry this suffix. 

\item \texttt{-monolexical-all} : all the occurrences of monolexical
  phrases are considered as term candidate occurrences

\item \texttt{-monolexical-included} : occurrences of monolexical term
  candidates that appear in complex term candidates are also
  considered as term candidate occurrences

\item \texttt{-termino} \textit{filename}: (optional) name of a file
  containing testified terms (in the TreeTagger format)

\item \texttt{-match-type} \textit{loose} or \textit{strict} : affects
  the matching of testified terms in the corpus; with \textit{loose},
  testified terms match either inflected or lemmatized forms of each
  word; with \textit{strict}, testified terms match the combination of
  inflected form and PoS tag of each word; by default, testified terms match match inflected forms of words

% \item \texttt{-correction}: (optional) allows the correction of PoS or
%   lemma information of the MNPs according to the testified terms provided.
\end{itemize}


\textbf{Options for result display} : 
\begin{itemize}

\item \texttt{-termList}: (optional) displays a list of terms and
  sub-terms along with their frequency in a text file. To extract only term candidates containing more than one word (multi-word term candidates), specify the option \"multi\". By default, all term candidates will be extracted, monolexical and multi-word term candidates

% \item \texttt{-basicTerms}: (optional) displays the list of basic term candidates (every pairs of words in a Head-Modifier relationship)


% \item \texttt{-MNPhtml}: (optional) displays \index{display} of the
%   corpus marked with MNPs in a HTML file. Information on
%   \begin{itemize}
%   \item the lexical form of each MNP;
%   \item the potential islands of reliability (exogenous and
%   indogenous) identified in the MNP;
% \item its parsing status (parsed or
%   not, parsing method applied)
% \item the potential corrections made
%   on the lexical information using the testified terms
%   \end{itemize}
%   can be seen by clicking on each occurrence of an MNP marked in the corpus. 


\item \texttt{-xmlout}: (optional) displays the term 
  candidates in XML format�. Term candidates are linked to each others
  in a syntactic network.

\item \texttt{-printChunking}: (optional) displays \index{display} of the
  corpus marked with extracted (\textit{i.e.} term candidates) and rejected phrases in a HTML file

% \item \texttt{-compareChunking} \textit{filename}: comparison of the chunking (MNP
%   identification) of a first version of the current analysed corpus (the option
%   -printChunking was specified when analysing the first corpus, here
%   specified as \textit{filename}: CORPUS\_Chunking\_corpus.html)

% \item \texttt{-heads}: (optional) selects the display of the terms where the 
%   syntactic head is marked
%   (see section \ref{sec:term-head-format}) \index{tete syntaxique\@t�te syntaxique}.

\item \texttt{-TT-for-BioLG} : annotation of the corpus with testified
  terms in a XML format compatible with the "biolg" version of the
  Link Grammar Parser
\item \texttt{-XML-corpus-for-BioLG} : creation of a BioLG compatible
  XML version of the corpus with PoS tags marked form each word
\item \texttt{-annotate-only} : only annotate testified terms (no acquisition)

\item \texttt{-TC-for-BioLG} : annotation of the corpus with term candidates in a XML format compatible with the "BioLG" version of the Link Grammar Parser

\item  \texttt{-TTG-style-term-candidates} : term candidates are displayed in TreeTagger output format. Term separator is the sentence boundary tag \"SENT\". To extract only term candidates containing more than one word (multi-word term candidates), specify the option \"multi\". By default, all term candidates will be extracted, monolexical and multi-word term candidates

\end{itemize}  


\subsection{File Hierarchy}

The package provides several directories and files. Configuration
files are use to define specific features for each language. 

\begin{itemize}
\item \verb+yatea+: main program
  % \item \verb+CORPUS+: r�pertoire de corpora (fichiers entr�e) 
  %   \begin{itemize}
  %   \item \verb+LANG+: directory containing the r�pertoire contenant les corpora pour une langue donn�e
  %   \end{itemize}
\item \verb+/share/YaTeA/config/+: directory containing the linguistic resources
  required for the term extraction and analysis. The resources are
  distributed according to the language (LANG).
  \begin{itemize}
  \item \verb+LANG+: directory including the resources for a given
    language (for instance : EN for English) . % We propose to use
    % the acronym defined in the norm ISO 3166 (related to the country code).
    \begin{itemize}
    \item \verb+TagSet+: the file contains the definition of the
      different classes of tags or inflected forms: \texttt{CANDIDATES},
      \texttt{DETERMINERS}, \texttt{PREPOSITIONS} and \texttt{COORDINATIONS}, used in the Parsing Patterns
    \item \verb+ChunkingExceptions+: the file provides the set
      of exceptions items regarding the chunking of the corpus into maximal
      noun phrases. Items can be PoS tags or words in their inflected
      or lemmatized form. 
    \item \verb+ChunkingFrontiers+: the file provides the set
      of items used as frontiers to the chunking of the corpus into maximal
      noun phrases. Items can be PoS tags or words in their inflected
      or lemmatized form. 
    \item \verb+CleaningFrontiers+: the file provides the
      set of items which cannot be used as frontiers of the maximal noun
      phrases. Items can be PoS tags or words in their inflected
      or lemmatized form.
    \item \verb+CleaningExceptions+: the file provides the set
      of exceptions items regarding the pruning of phrases.
    \item \verb+ForbiddenStructures+: the file contains
      structures that cannot appear in the noun phrases. Structures
      can mix PoS tags or words in their inflected or lemmatized form.
    \item \verb+Options+: options relating to linguistic processing
   % \item \verb+MiscellanousOptions+: other options
    \item \verb+ParsingPatterns+: the file contains the
      patterns defined to analyse the noun phrases.
    \end{itemize}
  \end{itemize}
  % \item \verb+TOOLS+ : r�pertoire d'outils de segmentation et de nettoyage des textes
% \item \verb+RESULTS+ : this directory contains the results of the term
%   extraction and analysis.
%   \begin{itemize}
%   \item \verb+LANG+: this directory is named according to the working
%     corpus language provided in command line (\textit{e.g.} \verb+EN+).
%     \begin{itemize}
%     \item \verb+CORPUS+ or \verb+CORPUS/SUFFIX+: a directory is
%       created for each analysed corpus. The directory name is the name
%       of the corpus minus its path and extension. If a suffix is
%       provided in command line, a sub-directory is created to store the
%       results of the current version of the corpus analysis. Results
%       are further organized according to the format (\textit{i.e.} extension) of the files:
%         \begin{itemize}
%         \item \verb+raw+: contains files in text format (or any other)
%         \item \verb+html+: contains files in HTML format
%         \item \verb+xml+: contains files in XML format
%         \end{itemize}
%       \end{itemize}
%     % Currently, this directory contains the following files (where
% %     \verb+<CORPUS>+ is the basename of the corpus without the final
% %     extension):
% %     \begin{itemize}
% %     \item \verb+<CORPUS>_MaximalNounPhrases_TYPE.txt+: list of the maximal
% %       noun phrases. TYPE is the type of information requested by the
% %       user in the command line option \texttt{-mnp}.

% %     \item \verb+<CORPUS>_UnparsedMNP.txt+: list of the noun phrases
% %       that could not be parsed.

% %     \item \verb+<CORPUS>_ParsedTerms.txt+: list of the parsed
% %       noun phrases.

% %     \end{itemize}

%   \end{itemize}
\item \verb+????+ : this directory contains the documentation for the
  term extractor \YaTeA. 
   \begin{itemize}
  %  \item \verb+DeliverableD6.3.pdf+: Description of the architecture
%      and implementation of \YaTeA. 
   \item \verb+Yatea_user_manual.pdf+: this document
     \item \verb+Penn-Treebank-English-Tagset.ps+: description of the
       tagset for English data 
     \item \verb+french-TreeTagger-tagset.html+: description of the
       tagset for French data 
   \end{itemize}
\end{itemize}


\section{Input formats}
\label{sec:input-format}

\subsection{Corpus}

As an input, the term extractor requires a corpus which has been
segmented in words and sentences, lemmatized and tagged with
part-of-speech (PoS)
information. Currently, the prototype takes as input the TreeTagger
ouput format. Information relating to a word is recorded in a
line. Each line contains three fields separated by a tabulation:
the inflected form, the PoS tag, and the lemma (see sample in 
Figure \ref{corpus}). If the lemma of the word appears as
$<$unknown$>$, its inflected form is used as a lemma.


\begin{figure}
\begin{verbatim}
However         RB      however
,               ,       ,
the             DT      the
concentration   NN      concentration
of              IN      of
protein         NN      protein
p6              NN      <unknown>
appeared        VVD     appear
to              TO      to
be              VB      be
important       JJ      important
for             IN      for
repression      NN      repression
of              IN      of
the             DT      the
early           JJ      early
promoter        NN      promoter
C2              NN      <unknown>
.               SENT    .
\end{verbatim}
  \caption{Sample of the input format}\label{corpus}
\end{figure}

\subsection{Resources}\label{ress}

The term extractor requires additional resources (chunking frontiers,
extraction patterns, \emph{etc}.) to process the input corpus. The directory
\texttt{/share/YaTeA/config/}\textit{LANG} contains the resources for a given
language, where \textit{LANG} is declared as an option in the
command line. % Each file
% related to a resource has the suffix \texttt{\_}\textit{LANG}. For
% instance, to process an English corpus, the language option is
% \texttt{-lang EN}. In that respect, the term extractor uses the resources
% in the directory \texttt{DATA/EN}, where each file ends with
% \texttt{\_EN} (\texttt{ParsingPatterns\_EN}).

The resources are provided as described below.

\subsubsection{TagSet}
\label{sec:tagset}
\texttt{TagSet} contains the definition for the tags allowing the
phrases to be parsed. It must conform to the following format (see
sample in Figure \ref{fig:tagset}):
 \begin{itemize}
  \item the comments start with the character \verb+#+.
  \item four kinds of tags are declared: ``CANDIDATES'',
    ``DETERMINERS'', ``PREPOSITIONS'' and  ``COORDINATIONS'' which
    list the tags (or inflected forms for prepositions) for
    respectively content  words, determiners, prepositions and coordination words. The line starts and ends with the characters
    \texttt{!!}. The tags are defined as regular expressions. Note
    that the ``coordination'' tags are not used in the current version
    of the package.
 \end{itemize}
\begin{figure}[!ht]
\begin{verbatim}
# Tags and words allowed in the parsing of the phrases
!!CANDIDATES = ((JJS)|(JJR)|(JJ)|(NNS)|(NN)|(NP)|(CD)|(VBG)|(FW)|(NPS))!!
!!DETERMINERS = ((DT))!!
!!PREPOSITIONS = ((of)|(to))!!
!!COORDINATIONS = ((CC))!!
\end{verbatim}
  
  \caption{Sample of the tagset definition file}\label{fig:tagset}
\end{figure}
  

\subsubsection{ParsingPatterns}
\label{sec:parsingpatterns}
\texttt{ParsingPatterns} contains the definition for the parsing
patterns used to parse phrases identified during the chunking step. It must conform to the following
  format (see sample in Figure \ref{fig:patterns}):
  

  \begin{itemize}
  \item the comments start with the character \verb+#+.
    
  
  \item the syntactic patterns are defined in a line with 3 fields
    separated by a tabulation. The first field describes the syntactic
    analysis of the matching term (\texttt{H} a head component of the
    term, and \texttt{M} for a modifier component). The two remaining fields
    give the priority of the pattern (integer) and the analysis
    direction (LEFT or RIGHT).
    The position of the most global head in the pattern determines the direction. 
    
  \end{itemize}
  \begin{figure}[!ht]
\begin{verbatim}
#Pattern                            Priority   Direction
( JJR<=M> NNS<=H> )                     1       RIGHT
( JJ<=M> ( JJ<=M> NNS<=H> )<=H> )       1       RIGHT
( NN<=H> of NNS<=M> )                   1       LEFT
( NN<=H> to DT NN<=M> )                 1       LEFT
\end{verbatim}

    \caption{Sample of the parsing pattern definition file}\label{fig:patterns}
  \end{figure}
  

  \subsubsection{ChunkingFrontiers}
  \label{sec:chunkingfrontiers}

  
 \texttt{ChunkingFrontiers} contains the declaration of the
 external frontiers of the chunk, \textit{i.e.} tags or words that
 cannot appear in a term (\textit{e.g.} verbs). They are used to chunk the corpus and identify the
  maximal phrases. The required format is given below:

  \begin{itemize}
  \item the comments start with the character \verb+#+.

  \item each line describes a chunking frontier according to two fields:
    the information which will be used as a frontier (inflected form --
    \texttt{IF}, part-of-speech tag -- \texttt{POS}, or lemma --
    \texttt{LF}), and the value of the frontier (see the example in 
    Figure \ref{fig:chunkingfrontiers}).
    \begin{figure}[!ht]
\begin{verbatim}
# Type  Value
POS     VB
POS     VV
POS     VBD
POS     IN
\end{verbatim}
      \caption{Sample of the chunking frontier file.}\label{fig:chunkingfrontiers}
    \end{figure}

  \end{itemize}


  \subsubsection{ChunkingExceptions}
  \label{sec:chunkingexceptions}

  
 \texttt{ChunkingExceptions} contains the declaration of the frontier exceptions.  They are used to prevent the corpus from being 
  chunked in some specific cases. The format is given below:

  \begin{itemize}
  \item the comments start with the character \verb+#+.

  \item each line describes a frontier exception according to two fields:
    the information which will be used as an exception (inflected form --
    \texttt{IF}, part-of-speech tag -- \texttt{POS}, or lemma --
    \texttt{LF}), and the value of the exception (see the example in
    figure \ref{fig:chunkingexceptions} where the inflected form ``of'' is an
    exception of the frontier tag ``IN'' declared in the file of chunking
    frontiers).
    \begin{figure}[!ht]
\begin{verbatim}
# Type of the element   Value of the element
IF     of
\end{verbatim}
      \caption{Sample of the exception frontier file}\label{fig:chunkingexceptions}
    \end{figure}

  \end{itemize}


  \subsubsection{CleaningFrontiers}
  \label{sec:cleaningfrontiers}

  
 \texttt{CleaningFrontiers} contains the declaration of the cleaning
  frontiers. They are used to improve the chunking
  by removing words (like determiners) that cannot appear in a start
  or final position of a term. 
  % 
  The file contains the list of the elements (words, tags, or lemma) that
  are allowed to start or end a maximal noun phrase. 

The format is given below:

  \begin{itemize}
  \item the comments start with the character \verb+#+.

  \item each line describes a cleaning frontier according to two fields:
    the information which will be used as a frontier (inflected form --
    \texttt{IF}, part-of-speech tag -- \texttt{POS}, or lemma --
    \texttt{LF}), and the value of the frontier (see the example in
    Figure \ref{fig:cleaningFrontiers}).
    \begin{figure}[!ht]
\begin{verbatim}
# Type  Value
POS     NN
POS     NNS
POS     NP
\end{verbatim}
      \caption{Sample of the cleaning frontier file}\label{fig:cleaningFrontiers}
    \end{figure}

  \end{itemize}

 \texttt{CleaningExceptions} contains the declaration of the cleaning
 exceptions.  They are used to prevent from the presence of specific
 words for which the PoS tag would be allowed on the borders of a phrase. The format is given below:

  \begin{itemize}
  \item the comments start with the character \verb+#+.

  \item each line describes a frontier exception according to two fields:
    the information which will be used as an exception (inflected form --
    \texttt{IF}, part-of-speech tag -- \texttt{POS}, or lemma --
    \texttt{LF}), and the value of the exception (see the example in
    figure \ref{fig:cleaningexceptions} where the inflected form ``of'' is an
    exception of the frontier tag ``IN'' declared in the file of chunking
    frontiers).
    \begin{figure}[!ht]
\begin{verbatim}
# Type   Value
IF     several
\end{verbatim}
      \caption{Sample of the cleaning exception file}\label{fig:cleaningexceptions}
    \end{figure}

  \end{itemize}

  \subsubsection{ForbiddenStructures}
  \label{sec:forbiddenstructures}

   \texttt{ForbiddenStructures} contains the declaration of the forbidden structures. They describe the word/tag sequences that cannot appear in a term. For instance, in ``\textit{in
    which \textbf{case the timers}}'', the noun phrase ``\textit{case the timers}'' has the
  PoS form \texttt{NN DT NNS}. To avoid such improper constructions,
  we do not allow the sequence \texttt{NN DT}. 
  Forbidden structures can contain a mix of inflected, lemmatized
  forms and tags.
  There are two possible operations when a forbidden structure is
  found: splitting or deletion, according to what is declared in the file.

The format is given below:

  \begin{itemize}
  \item the comments start with the character \verb+#+.

  \item each line describes a forbidden structure according to five
    fields (the last one is defined only for the splitting action):
     the value of the pattern with the type of
    information (inflected form --
    \texttt{IF}, part-of-speech tag -- \texttt{POS}, or lemma --
    \texttt{LF}), the position in the phrase where the sequences is expected to be
    found (START, END or ANY), the action
    (\texttt{split} or
    \texttt{delete}) for the structures in ANY position, and the
    position of the word, lemma or tag that will be used to perform
    the splitting (\textit{i.e.} pivot). Figure \ref{fig:forbiddenstructures} shows an example of a forbidden structure file.
    \begin{figure}[!ht]
\begin{verbatim}
# Value\Type    	Position	Action	Split after
JJ\POS of\LF     	START
JJR\POS	                END
such\LF	                ANY     	delete
of\LF DT\POS of\LF	ANY     	split	1
\end{verbatim}
      \caption{Sample of the forbidden structure file}\label{fig:forbiddenstructures}
    \end{figure}

  \end{itemize}



  \subsubsection{Options}
  \label{sec:options}

 \texttt{Options} contains additional configuration information. The file defines the options for
  the parsing of the terms for a given language. Each option is
  defined on a line as follows:
  \texttt{OPTION\_NAME = value}.

The mandatory options are the following:

\begin{itemize}
\item \texttt{DOCUMENT\_BOUNDARY}: a (PoS) tag in the corpus that indicates
  the limit between 2 documents
\item \texttt{SENTENCE\_BOUNDARY}: a (PoS) tag in the corpus that indicates
  the limit between 2 sentences
\item \texttt{PARSING\_DIRECTION}: the possible values of this option are 
  ``LEFT'' or ``RIGHT''. It defines the tendancy of the given language to
  have the head of a noun phrase on its left or right edge. For
  instance, the preferential direction of the analysis in English is
  ``RIGHT'', while for French it is ``LEFT''. This information is used
  to perform a priority sorting of the parsing patterns when several
  are possible in the prograssive parsing. This information
  refers to the information \textit{Direction} (second field) defined for
  each syntactic pattern in the file \texttt{ParsingPatterns}.


\item \texttt{COMPULSORY\_ITEM}: this option is declared
  as a regular expression, using disjonction (OR) and quantifiers
  (\texttt{+}, \texttt{?}, \texttt{*}). It is used during the chunking of
  the phrases to check that at least one compulsory tag appears
  in each extracted term candidate. For instance, in English, as we use the PennTree Bank tagset with the
  TreeTagger, the default option is the following: \texttt{COMPULSORY\_ITEM = (N.*)}. In that respect, we check that there is
  at least one common or proper noun in each identified phrase.
\item \texttt{PHRASE\_MAXIMUM\_LENGTH} defines the maximal number
  of tokens allowed in a
  maximal phrase. The default value is 12.

\item \texttt{COLOR\_BLIND} offers brighter colors in the HTML output files 
\end{itemize} 


% \subsubsection{MiscellaneousOptions}
% \label{sec:miscoptions}

%  \texttt{MiscellaneousOptions} contains the declaration of  miscellaneous options. It defines the processing
%   options according to the format \texttt{OptionName = value}:, 

% \begin{itemize}
% \item \texttt{COLOR\_BLIND} offers brighter colors in the HTML output files 
% \item \texttt{TAGGER} defines the path to the tagger used to tag plain
%   lists of testified terms
% \end{itemize} 

\subsubsection{Testified terms}
Lists of testified terms can be provided in % different formats:
a \textsc{tagged} % or \textsc{bracketed}.
format.


\begin{itemize}
\item 
A \textsc{tagged} format terminology is a file containing terms in the TreeTagger
format as shown in
Figure  \ref{TTtagged}. Terms are separated by the SENTENCE\_BOUNDARY
tag as defined in
the 
Options file.
\begin{figure}[!ht]
\begin{verbatim}
catabolite      NN      catabolite
repression      NN      repression
.               SENT    .
core            NN      core
RNA             NN      RNA
polymerase      NN      polymerase
.
\end{verbatim}
      \caption{Sample of a testified term file in \textsc{tagged} format}\label{TTtagged}
    \end{figure}

%   \item  
% A \textsc{bracketed} format terminology  is a file containing terms along with their
% syntactic analysis as shown in
% Figure  \ref{TTbracketed}. One term per line is allowed. 6 fields
% provide information on each testified: the bracketed parse, the
% inflected form, PoS sequence and lemmatized form of the term, its
% global syntactic head PoS tag, and the name of the resource it comes from
% \begin{figure}[!ht]
% \begin{small}
% \begin{verbatim}
% ( heat<=M=NN> shock<=H=NN> )  heat shock  NN NN  heat shock  NN  PhB
% ( spore<=M=JJ> coat<=H=NN> )  spore coat  JJ NN  spore coat  NN  PhB
% ( protein<=H=NN> of ( Bacillus<=H=NN> subtilis<=M=NN> )<=M=NN> )  
%   protein of Bacillus subtilis  NN of NN NN  protein of Bacillus 
%   subtilis  NN  PhB
% \end{verbatim}
% \end{small}
%       \caption{Sample of a testified term file in \textsc{bracketed} format}\label{TTbracketed}
%     \end{figure}

\end{itemize}


\section{Output formats}
\label{sec:output-format}

We describe the XML format only in this section. For information on
the other output formats, please see the option description in Section
\ref{RunningYatea}.

\subsubsection{xmlout option format}
\label{sec:xmloutformat}
The result of the extraction and analysis of term candidates is
displayed in XML format. % The XLM format output allows the display of
% non ambiguous term candidates only, \textit{i.e.} term candidates for which
% only one syntactic tree was found.
It contains all the extracted term candidates either corresponding to
complete phrases identified during the chunking step or to their
sub-parts processed during the parsing step. 
\begin{figure}
  \includegraphics[scale=1]{TC_tree}
  \caption{Example of term candidates processed from a phrase}
  \label{fig:TCtree}
\end{figure}


The DTD (Document Type Definition) that was defined for the XML
display appears in Appendix \ref{sec:dtd-xml-display}. Figure \ref{fig:exampleXML} shows an example of the XML output for a
term candidate.


\begin{figure}[!htbp]
  \centering
  {\lstinputlisting[firstline=267,lastline=301]{samples/pmid10788508-v2.genia2.xml}}
  \caption{Example of the XML display for a term candidate}
  \label{fig:exampleXML}
\end{figure}

\appendix
\section{DTD for the XML display}\label{sec:dtd-xml-display}

\lstinputlisting{DTD/yatea.dtd}

\end{document}
